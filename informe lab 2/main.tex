\documentclass[12pt,a4paper]{article}

% ----------------- Paquetes -----------------
\usepackage[spanish]{babel}
\usepackage[utf8]{inputenc}
\usepackage[T1]{fontenc}
\usepackage{amsmath, amssymb, graphicx}
\usepackage{float}
\usepackage{geometry}
\usepackage{caption}
\usepackage{booktabs}
\usepackage{url}
\usepackage[hidelinks]{hyperref}
\usepackage{siunitx}
\usepackage{setspace}
\usepackage{fancyhdr}

% ----------------- Configuración general -----------------
\geometry{margin=2.5cm}
\setstretch{1.3}
\pagestyle{fancy}
\fancyhf{}
\rhead{\thepage}
\lhead{Informe de Física II}
\renewcommand{\figurename}{Fig.}
\renewcommand{\tablename}{Tabla}

% ----------------- Datos del informe -----------------
\title{\textbf{Física 2 --- Laboratorio 2}\\[0.2cm]
\large Campo magnético terrestre}
\author{
Hector Pereira \\ hector.pereira@estudiantes.utec.edu.uy
}
\date{Fecha: \today}

% ============================================================
\begin{document}
\maketitle
\thispagestyle{empty}
\vspace{1cm}
% ----------------- Resumen -----------------
\begin{abstract}
En este resumen se redactan los aspectos generales del trabajo, los objetivos principales y las conclusiones más relevantes. El resumen debe ser autocontenido, sin referencias a ecuaciones o figuras.
\end{abstract}

\vspace{1cm}
\noindent\textbf{Palabras clave:} palabra1, palabra2, palabra3.

\newpage

% ----------------- Introducción -----------------
\section{Introducción}
Describir el interés físico del problema y su contexto.  
Indicar claramente los objetivos del trabajo y la motivación para realizar el experimento o simulación.  
Finalizar la introducción con una frase del tipo:  
\textit{“El objetivo de este trabajo es…”}.

% ----------------- Materiales y Métodos -----------------
\section{Materiales y Métodos}
Describir los datos experimentales o de simulación utilizados, el método de adquisición de los mismos, los instrumentos empleados y el procedimiento de análisis.  
Evitar describir comandos de software específicos; en su lugar, explicar el razonamiento matemático o físico que se aplica.

\subsection{Dispositivo experimental}
(Describir aquí el montaje, incluir esquemas o fotografías.)

\subsection{Tratamiento de datos}
(Describir los cálculos, ecuaciones y métodos usados.)

% ----------------- Resultados -----------------
\section{Resultados}
Presentar las tablas, figuras o gráficos con la numeración y leyenda adecuada.

% \begin{figure}[H]
%     \centering
%     \includegraphics[width=0.8\textwidth]{figura_ejemplo.png}
%     \caption{Datos experimentales obtenidos para el sistema analizado.}
%     \label{fig:datos}
% \end{figure}

\begin{table}[H]
\centering
\caption{Ejemplo de tabla de resultados}
\begin{tabular}{ccc}
\toprule
Variable & Valor & Unidad \\
\midrule
$t$ & 0.5 & s \\
$V$ & 2.1 & V \\
\bottomrule
\end{tabular}
\end{table}

% ----------------- Discusión y Conclusiones -----------------
\section{Discusión y Conclusiones}
Analizar los resultados, compararlos con referencias teóricas o bibliográficas, y discutir posibles fuentes de error.  
Finalizar con un párrafo de conclusiones claras que respondan a los objetivos planteados.

% ----------------- Bibliografía -----------------
\begin{thebibliography}{9}
\bibitem{serway}
Raymond A. Serway, John W. Jewett Jr. (2017). \textit{Física para ciencias e ingeniería}. México: Cengage Learning.

\bibitem{apa}
Normas APA. \url{http://www.cva.itesm.mx/biblioteca/pagina_con_formato_version_oct/apa.htm}
\end{thebibliography}

\end{document}
