\documentclass[conference]{IEEEtran}

% ==== Paquetes útiles (mínimos) ====
\usepackage[spanish,es-noquoting,es-tabla]{babel}
\usepackage[T1]{fontenc}
\usepackage[utf8]{inputenc}
\usepackage{graphicx}
\usepackage{amsmath,amssymb}
\usepackage{booktabs}
\usepackage{hyperref}
\usepackage{microtype}
\graphicspath{{figuras/}{imagenes/}{img/}}

% ==== Datos del artículo ====
\title{Título del Informe (estilo IEEE)}
\author{
  \IEEEauthorblockN{Nombre Apellido}
  \IEEEauthorblockA{Institución / Departamento\\
  Ciudad, País\\
  \texttt{correo@ejemplo.com}}
}

\begin{document}
\maketitle

\begin{abstract}
Breve resumen del trabajo (150–250 palabras).
\end{abstract}

\begin{IEEEkeywords}
palabra clave 1, palabra clave 2, palabra clave 3
\end{IEEEkeywords}


\input{1introduccion}
\input{2marcoTeorico}
\input{3metodologia}
\input{4resultados}
\input{5conclusiones}

% ==== Bibliografía ====
\bibliographystyle{IEEEtran}
\nocite{*}
% \bibliography{6referencias}

% ==== Anexos ====
\appendices
\input{7anexos}

\end{document}